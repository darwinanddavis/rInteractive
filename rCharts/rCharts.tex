\documentclass[10,portrait]{article}
\usepackage{lmodern}
\usepackage{amssymb,amsmath}
\usepackage{ifxetex,ifluatex}
\usepackage{fixltx2e} % provides \textsubscript
\ifnum 0\ifxetex 1\fi\ifluatex 1\fi=0 % if pdftex
  \usepackage[T1]{fontenc}
  \usepackage[utf8]{inputenc}
\else % if luatex or xelatex
  \ifxetex
    \usepackage{mathspec}
  \else
    \usepackage{fontspec}
  \fi
  \defaultfontfeatures{Ligatures=TeX,Scale=MatchLowercase}
\fi
% use upquote if available, for straight quotes in verbatim environments
\IfFileExists{upquote.sty}{\usepackage{upquote}}{}
% use microtype if available
\IfFileExists{microtype.sty}{%
\usepackage[]{microtype}
\UseMicrotypeSet[protrusion]{basicmath} % disable protrusion for tt fonts
}{}
\PassOptionsToPackage{hyphens}{url} % url is loaded by hyperref
\usepackage[unicode=true]{hyperref}
\PassOptionsToPackage{usenames,dvipsnames}{color} % color is loaded by hyperref
\hypersetup{
            pdftitle={Interactive plots using R},
            colorlinks=true,
            linkcolor=pink,
            citecolor=red,
            urlcolor=blue,
            breaklinks=true}
\urlstyle{same}  % don't use monospace font for urls
\usepackage[margin=1in]{geometry}
\usepackage[]{biblatex}
\usepackage{color}
\usepackage{fancyvrb}
\newcommand{\VerbBar}{|}
\newcommand{\VERB}{\Verb[commandchars=\\\{\}]}
\DefineVerbatimEnvironment{Highlighting}{Verbatim}{commandchars=\\\{\}}
% Add ',fontsize=\small' for more characters per line
\usepackage{framed}
\definecolor{shadecolor}{RGB}{248,248,248}
\newenvironment{Shaded}{\begin{snugshade}}{\end{snugshade}}
\newcommand{\KeywordTok}[1]{\textcolor[rgb]{0.13,0.29,0.53}{\textbf{#1}}}
\newcommand{\DataTypeTok}[1]{\textcolor[rgb]{0.13,0.29,0.53}{#1}}
\newcommand{\DecValTok}[1]{\textcolor[rgb]{0.00,0.00,0.81}{#1}}
\newcommand{\BaseNTok}[1]{\textcolor[rgb]{0.00,0.00,0.81}{#1}}
\newcommand{\FloatTok}[1]{\textcolor[rgb]{0.00,0.00,0.81}{#1}}
\newcommand{\ConstantTok}[1]{\textcolor[rgb]{0.00,0.00,0.00}{#1}}
\newcommand{\CharTok}[1]{\textcolor[rgb]{0.31,0.60,0.02}{#1}}
\newcommand{\SpecialCharTok}[1]{\textcolor[rgb]{0.00,0.00,0.00}{#1}}
\newcommand{\StringTok}[1]{\textcolor[rgb]{0.31,0.60,0.02}{#1}}
\newcommand{\VerbatimStringTok}[1]{\textcolor[rgb]{0.31,0.60,0.02}{#1}}
\newcommand{\SpecialStringTok}[1]{\textcolor[rgb]{0.31,0.60,0.02}{#1}}
\newcommand{\ImportTok}[1]{#1}
\newcommand{\CommentTok}[1]{\textcolor[rgb]{0.56,0.35,0.01}{\textit{#1}}}
\newcommand{\DocumentationTok}[1]{\textcolor[rgb]{0.56,0.35,0.01}{\textbf{\textit{#1}}}}
\newcommand{\AnnotationTok}[1]{\textcolor[rgb]{0.56,0.35,0.01}{\textbf{\textit{#1}}}}
\newcommand{\CommentVarTok}[1]{\textcolor[rgb]{0.56,0.35,0.01}{\textbf{\textit{#1}}}}
\newcommand{\OtherTok}[1]{\textcolor[rgb]{0.56,0.35,0.01}{#1}}
\newcommand{\FunctionTok}[1]{\textcolor[rgb]{0.00,0.00,0.00}{#1}}
\newcommand{\VariableTok}[1]{\textcolor[rgb]{0.00,0.00,0.00}{#1}}
\newcommand{\ControlFlowTok}[1]{\textcolor[rgb]{0.13,0.29,0.53}{\textbf{#1}}}
\newcommand{\OperatorTok}[1]{\textcolor[rgb]{0.81,0.36,0.00}{\textbf{#1}}}
\newcommand{\BuiltInTok}[1]{#1}
\newcommand{\ExtensionTok}[1]{#1}
\newcommand{\PreprocessorTok}[1]{\textcolor[rgb]{0.56,0.35,0.01}{\textit{#1}}}
\newcommand{\AttributeTok}[1]{\textcolor[rgb]{0.77,0.63,0.00}{#1}}
\newcommand{\RegionMarkerTok}[1]{#1}
\newcommand{\InformationTok}[1]{\textcolor[rgb]{0.56,0.35,0.01}{\textbf{\textit{#1}}}}
\newcommand{\WarningTok}[1]{\textcolor[rgb]{0.56,0.35,0.01}{\textbf{\textit{#1}}}}
\newcommand{\AlertTok}[1]{\textcolor[rgb]{0.94,0.16,0.16}{#1}}
\newcommand{\ErrorTok}[1]{\textcolor[rgb]{0.64,0.00,0.00}{\textbf{#1}}}
\newcommand{\NormalTok}[1]{#1}
\IfFileExists{parskip.sty}{%
\usepackage{parskip}
}{% else
\setlength{\parindent}{0pt}
\setlength{\parskip}{6pt plus 2pt minus 1pt}
}
\setlength{\emergencystretch}{3em}  % prevent overfull lines
\providecommand{\tightlist}{%
  \setlength{\itemsep}{0pt}\setlength{\parskip}{0pt}}
\setcounter{secnumdepth}{0}
% Redefines (sub)paragraphs to behave more like sections
\ifx\paragraph\undefined\else
\let\oldparagraph\paragraph
\renewcommand{\paragraph}[1]{\oldparagraph{#1}\mbox{}}
\fi
\ifx\subparagraph\undefined\else
\let\oldsubparagraph\subparagraph
\renewcommand{\subparagraph}[1]{\oldsubparagraph{#1}\mbox{}}
\fi

% set default figure placement to htbp
\makeatletter
\def\fps@figure{htbp}
\makeatother


\title{Interactive plots using R}
\author{Matthew Malishev\textsuperscript{1}*\\
\emph{\textsuperscript{1} Department of Biology, Emory University, 1510
Clifton Road NE, Atlanta, GA, USA, 30322}}
\date{}

\begin{document}
\maketitle

{
\hypersetup{linkcolor=black}
\setcounter{tocdepth}{4}
\tableofcontents
}
~

Date: 2018-09-27\\
R version: 3.5.0\\
*Corresponding author:
\href{mailto:matthew.malishev@gmail.com}{\nolinkurl{matthew.malishev@gmail.com}}\\
This document can be found at
\url{https://github.com/darwinanddavis/rInteractive}

\newpage  

\subsection{Overview}\label{overview}

Interactive plots using javascript viz, including rCharts, d3js,
leaflet, Richshaw, to name a few. \texttt{rCharts} uses multiple
javascript plotting libraries.\\
\#\#\# Install dependencies

\subsubsection{Set plotting function}\label{set-plotting-function}

\begin{Shaded}
\begin{Highlighting}[]
\KeywordTok{require}\NormalTok{(}\StringTok{"RCurl"}\NormalTok{)}
\NormalTok{script <-}\StringTok{ }\KeywordTok{getURL}\NormalTok{(}\StringTok{"https://raw.githubusercontent.com/darwinanddavis/plot_it/master/plot_it.R"}\NormalTok{, }\DataTypeTok{ssl.verifypeer =} \OtherTok{FALSE}\NormalTok{)}
\KeywordTok{eval}\NormalTok{(}\KeywordTok{parse}\NormalTok{(}\DataTypeTok{text =}\NormalTok{ script))}

\KeywordTok{cat}\NormalTok{(}\StringTok{"plot_it( }\CharTok{\textbackslash{}n}\StringTok{0 for presentation, 1 for manuscript, }\CharTok{\textbackslash{}n}\StringTok{set colour for background, }\CharTok{\textbackslash{}n}\StringTok{set colour palette. use 'display.brewer.all()', }\CharTok{\textbackslash{}n}\StringTok{set alpha for colour transperancy, }\CharTok{\textbackslash{}n}\StringTok{set font style }\CharTok{\textbackslash{}n}\StringTok{)"}\NormalTok{)}
\KeywordTok{plot_it}\NormalTok{(}\DecValTok{0}\NormalTok{,}\StringTok{"blue"}\NormalTok{,}\StringTok{"Spectral"}\NormalTok{,}\StringTok{"Greens"}\NormalTok{,}\DecValTok{1}\NormalTok{,}\StringTok{"mono"}\NormalTok{) }\CommentTok{# set col function params}
\KeywordTok{plot_it_gg}\NormalTok{(}\StringTok{"white"}\NormalTok{) }\CommentTok{# same as above }
\end{Highlighting}
\end{Shaded}

\subsubsection{\texorpdfstring{\texttt{rCharts}}{rCharts}}\label{rcharts}

\paragraph{\texorpdfstring{\texttt{rCharts} plot
functions}{rCharts plot functions}}\label{rcharts-plot-functions}

\begin{Shaded}
\begin{Highlighting}[]
\NormalTok{types <-}\StringTok{ }\KeywordTok{c}\NormalTok{(}\StringTok{"points"}\NormalTok{,}\StringTok{"line"}\NormalTok{,}\StringTok{"bar"}\NormalTok{);types}
\end{Highlighting}
\end{Shaded}

\begin{verbatim}
[1] "points" "line"   "bar"   
\end{verbatim}

\paragraph{Scatterplot}\label{scatterplot}

\begin{Shaded}
\begin{Highlighting}[]
\KeywordTok{require}\NormalTok{(rCharts)}
\NormalTok{## Example 1 Facetted Scatterplot}
\NormalTok{data <-}\StringTok{ }\NormalTok{iris}
\KeywordTok{names}\NormalTok{(iris) =}\StringTok{ }\KeywordTok{gsub}\NormalTok{(}\StringTok{"}\CharTok{\textbackslash{}\textbackslash{}}\StringTok{."}\NormalTok{, }\StringTok{""}\NormalTok{, }\KeywordTok{names}\NormalTok{(iris))}
\NormalTok{r1 <-}\StringTok{ }\KeywordTok{rPlot}\NormalTok{(data}\OperatorTok{$}\NormalTok{SepalLength}\OperatorTok{~}\NormalTok{data}\OperatorTok{$}\NormalTok{SepalWidth, }\DataTypeTok{data=}\NormalTok{data, }\DataTypeTok{color=}\StringTok{'Species'}\NormalTok{, }\DataTypeTok{type=}\StringTok{'point'}\NormalTok{)}
\CommentTok{# r1$show('iframesrc', cdn  = TRUE) # chart shows up on Rpubs but not in Rmd preview }
\end{Highlighting}
\end{Shaded}

\paragraph{Barplot}\label{barplot}

\subsubsection{Polychart}\label{polychart}

\href{https://github.com/Polychart/polychart2}{Polychart}

Adding javascript

\begin{Shaded}
\begin{Highlighting}[]
\VariableTok{graph_chart1}\NormalTok{.}\AttributeTok{addHandler}\NormalTok{(}\KeywordTok{function}\NormalTok{(type}\OperatorTok{,}\NormalTok{ e)}\OperatorTok{\{}
  \KeywordTok{var}\NormalTok{ data }\OperatorTok{=} \VariableTok{e}\NormalTok{.}\AttributeTok{evtData}\OperatorTok{;}
  \ControlFlowTok{if}\NormalTok{ (type }\OperatorTok{===} \StringTok{'click'}\NormalTok{)}\OperatorTok{\{}
    \ControlFlowTok{return} \AttributeTok{alert}\NormalTok{(}\StringTok{"You clicked on car with mpg: "} \OperatorTok{+} \VariableTok{data}\NormalTok{.}\VariableTok{mpg}\NormalTok{.}\AttributeTok{in}\NormalTok{[}\DecValTok{0}\NormalTok{])}\OperatorTok{;}
  \OperatorTok{\}}
\OperatorTok{\}}\NormalTok{)}
\end{Highlighting}
\end{Shaded}

\subsubsection{MorrisJS}\label{morrisjs}

Interactive time series plot
\href{https://github.com/oesmith/morris.js}{with Morris JS}

\subsubsection{Rickshaw}\label{rickshaw}

Interactive time series with
\href{http://rpubs.com/Koba/80208}{Rickshaw}. See online example
\href{http://timelyportfolio.github.io/rCharts_rickshaw_gettingstarted/}{here}
and troubleshooting \href{http://rpubs.com/Koba/80208}{here}.

\begin{Shaded}
\begin{Highlighting}[]
\KeywordTok{require}\NormalTok{(slidify)}
\KeywordTok{require}\NormalTok{(rCharts)}
\end{Highlighting}
\end{Shaded}

Troubleshooting example.

\subsubsection{My data examples}\label{my-data-examples}

Examples of packages using my own data

\subsubsection{\texorpdfstring{Other interactive plotting libraries in
\texttt{R}}{Other interactive plotting libraries in R}}\label{other-interactive-plotting-libraries-in-r}

\begin{itemize}
\tightlist
\item
  \href{https://github.com/nachocab/clickme}{Clickme}\\
\item
  \href{http://rpubs.com/Koba/80208}{Rickshaw}
\end{itemize}

\printbibliography

\end{document}
